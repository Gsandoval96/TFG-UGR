\section{Introducción}

Este proyecto se centra en la investigación y aprendizaje sobre generación procedimental de contenidos (\acrshort{pcg} por su sigla en inglés, \acrlong{pcg}), con el fin de estudiar la viabilidad de aplicación de dicha técnica a la reimplementación de videojuegos clásicos. Además, se ha planteado como objetivo la utilización de diversos métodos de \acrshort{pcg} en la implementación de uno de estos videojuegos.\\

Junto con este objetivo principal, dicha implementación se afrontará sin recurrir a un motor de videojuegos, como podría ser Unity, Unreal Engine o Godot, y usando exclusivamente la biblioteca de alto nivel para gráficos \acrshort{3d} en JavaScript, Three.js \cite{three.js}.\\

Durante el desarrollo de este proyecto se requerirá aplicar los conocimientos adquiridos en el Grado en Ingeniería Informática, haciendo especial mención a las asignaturas \textit{``Informática Gráfica''} y \textit{``Sistemas Gráficos''} ya que abarcan el grueso del desarrollo del videojuego.\\

A continuación se muestra la motivación y objetivos del trabajo, continuando con una sección con la planificación realizada, una descripción de los recursos a utilizar, la metodología que se seguirá y la viabilidad del proyecto. Seguiremos con una sección en la que definiremos la generación procedimental de contenidos y haremos un recorrido histórico por su uso en los videojuegos, clasificando el contenido generado, los métodos empleado para generarlo y estableceremos una serie de propiedades deseables asociadas a la generación procedimental de contenido. A continuación, analizaremos el juego clásico escogido y haremos una propuesta de elementos a los que aplicar generación procedimental. Seguiremos con una breve sección de diseño que incluye diagramas de clase y modelos jerárquicos. Más adelante entraremos de lleno en la implementación de juego, haciendo breves menciones a los apartados más generales de funcionamiento del propio juego y centrándonos en la generación procedimental de contenido. Finalmente encontramos una sección que incluye las conclusiones alcanzadas con este trabajo y posibles trabajos futuros fruto del proyecto realizado. Además se incluye una sección con las referencias utilizadas y dos anexos, uno que incluye un manual de usuario y otro con un pequeño glosario.

% ---------------------------------------------------------------------------- %

\subsection{Motivación}

Este proyecto está motivado por el interés por aprender sobre la generación procedimental de contenido aplicada a videojuegos combinado con la afición y admiración por los videojuegos clásicos originarios de las máquinas arcade de los años 80. Por ello se ha propuesto un proyecto que combina ambos intereses, con el fin de aplicar técnicas de \acrshort{pcg} a videojuegos que originalmente no plantearon el uso de las mismas.\\

Además, con el fin de mejorar el aprendizaje sobre desarrollo de videojuegos a más bajo nivel, se ha planteado el uso de una biblioteca gráfica, concretamente Three.js \cite{three.js}, frente a un motor de videojuegos como Unity \cite{unity}.

% ---------------------------------------------------------------------------- %

\subsection{Objetivos}

Con las bases del proyecto establecidas, se plantea como objetivo general:

\begin{itemize}
    \item La investigación y el autoaprendizaje sobre generación procedimental de contenidos.
\end{itemize}

Por otro lado, se desarrollaran los siguientes objetivos específicos:

\begin{itemize}
    \item La aplicación de generación procedimental de contenidos a videojuegos clásicos, aplicando diversas técnicas.
    \item La reimplementación del videojuego clásico Pac-Man utilizando técnicas de bajo nivel y sin la ayuda de un motor de videojuegos.
\end{itemize}

% ---------------------------------------------------------------------------- %