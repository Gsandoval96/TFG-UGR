\section{Conclusiones y trabajos futuros}

Para terminar esta memoria haremos un reflexión en lo que se refiere a generación procedimental de contenido en el ámbito de los videojuegos como al aprendizaje realizado y conocimiento adquirido durante el desarrollo de este proyecto. Además, valoraremos los resultados obtenidos fruto de este proyecto, haciendo especial hincapié en lo logrado al aplicar generación procedimental a un videojuego clásico.\\

En primer lugar debemos hablar obligatoriamente del conocimiento adquirido a lo largo del desarrollo de este proyecto, partiendo de no conocer la generación procedimental de contenidos más allá de ser un concepto lejano y difuso, y llegando a tras estas 30 semanas, haber leído multitud de artículos e información relacionada con generación procedimental; haber entrado de lleno en su uso en el sector de los videojuegos, siendo capaz de establecer una clasificación del contenido generado, de los métodos empleados y de las propiedades deseables en estos. Todo ello culminando en la aplicación de técnicas de generación procedimental a un videojuego clásico como el escogido, Pac-Man.\\

Ya entrando en el proyecto específico, se ha logrado una reproducción fiel del videojuego escogido, aportando un enfoque personal y adaptándolo a un entorno en el que la responsabilidad recae principalmente en el programador y que ha permitido precisamente afrontar los conflictos propios de un entorno a más bajo nivel que el que puede proporcionar un motor de videojuegos. Precisamente la elección de Three.js para el desarrollo de este proyecto me ha permitido seguir fortaleciendo conocimientos adquiridos durante los estudios de Grado, algunos tan imprescindibles como puede ser la importancia de una gestión adecuada de la memoria.\\

Todo esto ha servido como base para desarrollar un algoritmo de generación procedimental de laberintos para el videojuego Pac-Man y que ha sido complementado con la generación procedimental de texturas para los muros de los laberintos y una mejora progresiva de la dificultad intrínseca a los comportamientos de la inteligencia artificial. Debemos destacar que las técnicas aplicadas caen directamente en la categoría de métodos tradicionales y generan contenido de varios niveles de la clasificación establecida, como son las texturas que pertenecen a los \textit{Game Bits} o los laberintos y el escenario que pertenecen al \textit{Game Space} y los \textit{Game Scenarios} relativamente.\\

En concreto para la generación de laberintos se ha diseñado un algoritmo que cumple prácticamente con la totalidad de las propiedades deseables para métodos de generación procedimental de contenido. El algoritmo genera niveles del juego en décimas de segundo, el contenido generado es de calidad ya que siempre genera laberintos válidos y jugables, sumado a la inclusión de las texturas, ofrece una variedad casi infinita de mapas distintos, siendo además ampliable y escalable simplemente añadiendo un mayor número de piezas estilo Tetris al conjunto utilizado, y finalmente, todo este contenido imita con facilidad el contenido generado por humanos, ofreciendo credibilidad.\\

Aún con estos logros, el proyecto ha abierto las puertas a otros tantos retos e inquietudes que perseguir en el futuro, tanto a nivel de mejora del propio proyecto software como a nivel de continuar el aprendizaje sobre generación procedimental.\\ 

Por una parte y respecto al proyecto podríamos destacar la mejora a nivel de aplicación, evolucionando de una herramienta cuyo propósito principal es mostrar las posibilidades que nos abre la generación procedimental de contenido a un videojuego propiamente dicho, con menús de opciones, una mejor progresión de dificultad o una inteligencia artificial aún más pulida.\\

Por otra parte, el proyecto no ha hecho más que abrirme los ojos ante el mundo de posibilidades que ofrece la generación procedimental de contenido. Por un lado, sería interesante explorar la posibilidad de combinar el algoritmo diseñado con métodos basados en búsqueda, permitiendo no solo crear infinidad de laberintos sino también clasificar y puntuar los mismos, permitiendo utilizar laberintos más sencillos en los primeros niveles e ir escalando la dificultad en lo que a distribución del escenario se refiere. Por otro lado, me planteo como proyecto futuro seguir investigando y aprendiendo sobre generación procedimental de contenido y en particular me propongo como objetivo próximo aprender sobre el algoritmo de colapso de la función de onda, ya que me ha resultado fascinante los resultados que produce.\\

Concluyamos remarcando el potencial que representa la generación procedimental de contenidos en el ámbito de los videojuegos, no solo permitiéndonos crear una cantidad prácticamente infinita de contenido de calidad sino que nos aporta un nuevo enfoque de cara al desarrollo y diseño de videojuegos.