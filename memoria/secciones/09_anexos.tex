
\phantomsection
\markboth{Anexo I: Manual de usuario}{Anexo I: Manual de usuario}
\addcontentsline{toc}{section}{Anexo I: Manual de usuario}
\section*{Anexo I: Manual de usuario}

\phantomsection
\addcontentsline{toc}{subsection}{Descarga e instalación}
\subsection*{Descarga e instalación}

El código fuente del proyecto software implementado para el trabajo fin de grado \textit{``Reimplementación de videojuegos clásicos en Three.js utilizando técnicas de generación procedimental''} se encuentra disponible en: 	\href{https://github.com/Gsandoval96/TFG-UGR}{https://github.com/Gsandoval96/TFG-UGR}.\\

Para ejecutar el proyecto se puede:

\begin{itemize}
    \item \textbf{En local:} descargar el proyecto, entrar desde terminal en la carpeta \textit{pacman} y ejecutar el comando \textit{python -m http.server} y abrir en el navegador la dirección \href{http://localhost:8000/}{http://localhost:8000/}. Una vez dentro, entraremos en la carpeta \textit{src}.
    \item \textbf{Online:} entrando en \href{https://gsandoval96.github.io/TFG-UGR/pacman/src/}{https://gsandoval96.github.io/TFG-UGR/pacman/src/}.
\end{itemize}

\phantomsection
\addcontentsline{toc}{subsection}{Controles}
\subsection*{Controles}

La aplicación se divide en dos pantallas. La pantalla de inicio que se maneja con el ratón y cuya única funcionalidad es comenzar la partida haciendo click izquierdo en el botón \textit{PLAY}. La pantalla de juego solo requiere como controles las flechas del teclado. Al ser una cámara en tercera persona y elevada, cada una de las flechas nos permitirá girar hacia la dirección correspondiente.

% ---------------------------------------------------------------------------- %

\newpage

\printglossary[title=Anexo II: Glosario, type=\acronymtype]
